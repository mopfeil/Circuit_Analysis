
% Default to the notebook output style

    


% Inherit from the specified cell style.




    
\documentclass[11pt]{article}

    
    
    \usepackage[T1]{fontenc}
    % Nicer default font (+ math font) than Computer Modern for most use cases
    \usepackage{mathpazo}

    % Basic figure setup, for now with no caption control since it's done
    % automatically by Pandoc (which extracts ![](path) syntax from Markdown).
    \usepackage{graphicx}
    % We will generate all images so they have a width \maxwidth. This means
    % that they will get their normal width if they fit onto the page, but
    % are scaled down if they would overflow the margins.
    \makeatletter
    \def\maxwidth{\ifdim\Gin@nat@width>\linewidth\linewidth
    \else\Gin@nat@width\fi}
    \makeatother
    \let\Oldincludegraphics\includegraphics
    % Set max figure width to be 80% of text width, for now hardcoded.
    \renewcommand{\includegraphics}[1]{\Oldincludegraphics[width=.8\maxwidth]{#1}}
    % Ensure that by default, figures have no caption (until we provide a
    % proper Figure object with a Caption API and a way to capture that
    % in the conversion process - todo).
    \usepackage{caption}
    \DeclareCaptionLabelFormat{nolabel}{}
    \captionsetup{labelformat=nolabel}

    \usepackage{adjustbox} % Used to constrain images to a maximum size 
    \usepackage{xcolor} % Allow colors to be defined
    \usepackage{enumerate} % Needed for markdown enumerations to work
    \usepackage{geometry} % Used to adjust the document margins
    \usepackage{amsmath} % Equations
    \usepackage{amssymb} % Equations
    \usepackage{textcomp} % defines textquotesingle
    % Hack from http://tex.stackexchange.com/a/47451/13684:
    \AtBeginDocument{%
        \def\PYZsq{\textquotesingle}% Upright quotes in Pygmentized code
    }
    \usepackage{upquote} % Upright quotes for verbatim code
    \usepackage{eurosym} % defines \euro
    \usepackage[mathletters]{ucs} % Extended unicode (utf-8) support
    \usepackage[utf8x]{inputenc} % Allow utf-8 characters in the tex document
    \usepackage{fancyvrb} % verbatim replacement that allows latex
    \usepackage{grffile} % extends the file name processing of package graphics 
                         % to support a larger range 
    % The hyperref package gives us a pdf with properly built
    % internal navigation ('pdf bookmarks' for the table of contents,
    % internal cross-reference links, web links for URLs, etc.)
    \usepackage{hyperref}
    \usepackage{longtable} % longtable support required by pandoc >1.10
    \usepackage{booktabs}  % table support for pandoc > 1.12.2
    \usepackage[inline]{enumitem} % IRkernel/repr support (it uses the enumerate* environment)
    \usepackage[normalem]{ulem} % ulem is needed to support strikethroughs (\sout)
                                % normalem makes italics be italics, not underlines
    

    
    
    % Colors for the hyperref package
    \definecolor{urlcolor}{rgb}{0,.145,.698}
    \definecolor{linkcolor}{rgb}{.71,0.21,0.01}
    \definecolor{citecolor}{rgb}{.12,.54,.11}

    % ANSI colors
    \definecolor{ansi-black}{HTML}{3E424D}
    \definecolor{ansi-black-intense}{HTML}{282C36}
    \definecolor{ansi-red}{HTML}{E75C58}
    \definecolor{ansi-red-intense}{HTML}{B22B31}
    \definecolor{ansi-green}{HTML}{00A250}
    \definecolor{ansi-green-intense}{HTML}{007427}
    \definecolor{ansi-yellow}{HTML}{DDB62B}
    \definecolor{ansi-yellow-intense}{HTML}{B27D12}
    \definecolor{ansi-blue}{HTML}{208FFB}
    \definecolor{ansi-blue-intense}{HTML}{0065CA}
    \definecolor{ansi-magenta}{HTML}{D160C4}
    \definecolor{ansi-magenta-intense}{HTML}{A03196}
    \definecolor{ansi-cyan}{HTML}{60C6C8}
    \definecolor{ansi-cyan-intense}{HTML}{258F8F}
    \definecolor{ansi-white}{HTML}{C5C1B4}
    \definecolor{ansi-white-intense}{HTML}{A1A6B2}

    % commands and environments needed by pandoc snippets
    % extracted from the output of `pandoc -s`
    \providecommand{\tightlist}{%
      \setlength{\itemsep}{0pt}\setlength{\parskip}{0pt}}
    \DefineVerbatimEnvironment{Highlighting}{Verbatim}{commandchars=\\\{\}}
    % Add ',fontsize=\small' for more characters per line
    \newenvironment{Shaded}{}{}
    \newcommand{\KeywordTok}[1]{\textcolor[rgb]{0.00,0.44,0.13}{\textbf{{#1}}}}
    \newcommand{\DataTypeTok}[1]{\textcolor[rgb]{0.56,0.13,0.00}{{#1}}}
    \newcommand{\DecValTok}[1]{\textcolor[rgb]{0.25,0.63,0.44}{{#1}}}
    \newcommand{\BaseNTok}[1]{\textcolor[rgb]{0.25,0.63,0.44}{{#1}}}
    \newcommand{\FloatTok}[1]{\textcolor[rgb]{0.25,0.63,0.44}{{#1}}}
    \newcommand{\CharTok}[1]{\textcolor[rgb]{0.25,0.44,0.63}{{#1}}}
    \newcommand{\StringTok}[1]{\textcolor[rgb]{0.25,0.44,0.63}{{#1}}}
    \newcommand{\CommentTok}[1]{\textcolor[rgb]{0.38,0.63,0.69}{\textit{{#1}}}}
    \newcommand{\OtherTok}[1]{\textcolor[rgb]{0.00,0.44,0.13}{{#1}}}
    \newcommand{\AlertTok}[1]{\textcolor[rgb]{1.00,0.00,0.00}{\textbf{{#1}}}}
    \newcommand{\FunctionTok}[1]{\textcolor[rgb]{0.02,0.16,0.49}{{#1}}}
    \newcommand{\RegionMarkerTok}[1]{{#1}}
    \newcommand{\ErrorTok}[1]{\textcolor[rgb]{1.00,0.00,0.00}{\textbf{{#1}}}}
    \newcommand{\NormalTok}[1]{{#1}}
    
    % Additional commands for more recent versions of Pandoc
    \newcommand{\ConstantTok}[1]{\textcolor[rgb]{0.53,0.00,0.00}{{#1}}}
    \newcommand{\SpecialCharTok}[1]{\textcolor[rgb]{0.25,0.44,0.63}{{#1}}}
    \newcommand{\VerbatimStringTok}[1]{\textcolor[rgb]{0.25,0.44,0.63}{{#1}}}
    \newcommand{\SpecialStringTok}[1]{\textcolor[rgb]{0.73,0.40,0.53}{{#1}}}
    \newcommand{\ImportTok}[1]{{#1}}
    \newcommand{\DocumentationTok}[1]{\textcolor[rgb]{0.73,0.13,0.13}{\textit{{#1}}}}
    \newcommand{\AnnotationTok}[1]{\textcolor[rgb]{0.38,0.63,0.69}{\textbf{\textit{{#1}}}}}
    \newcommand{\CommentVarTok}[1]{\textcolor[rgb]{0.38,0.63,0.69}{\textbf{\textit{{#1}}}}}
    \newcommand{\VariableTok}[1]{\textcolor[rgb]{0.10,0.09,0.49}{{#1}}}
    \newcommand{\ControlFlowTok}[1]{\textcolor[rgb]{0.00,0.44,0.13}{\textbf{{#1}}}}
    \newcommand{\OperatorTok}[1]{\textcolor[rgb]{0.40,0.40,0.40}{{#1}}}
    \newcommand{\BuiltInTok}[1]{{#1}}
    \newcommand{\ExtensionTok}[1]{{#1}}
    \newcommand{\PreprocessorTok}[1]{\textcolor[rgb]{0.74,0.48,0.00}{{#1}}}
    \newcommand{\AttributeTok}[1]{\textcolor[rgb]{0.49,0.56,0.16}{{#1}}}
    \newcommand{\InformationTok}[1]{\textcolor[rgb]{0.38,0.63,0.69}{\textbf{\textit{{#1}}}}}
    \newcommand{\WarningTok}[1]{\textcolor[rgb]{0.38,0.63,0.69}{\textbf{\textit{{#1}}}}}
    
    
    % Define a nice break command that doesn't care if a line doesn't already
    % exist.
    \def\br{\hspace*{\fill} \\* }
    % Math Jax compatability definitions
    \def\gt{>}
    \def\lt{<}
    % Document parameters
    \title{BodePlots}
    
    
    

    % Pygments definitions
    
\makeatletter
\def\PY@reset{\let\PY@it=\relax \let\PY@bf=\relax%
    \let\PY@ul=\relax \let\PY@tc=\relax%
    \let\PY@bc=\relax \let\PY@ff=\relax}
\def\PY@tok#1{\csname PY@tok@#1\endcsname}
\def\PY@toks#1+{\ifx\relax#1\empty\else%
    \PY@tok{#1}\expandafter\PY@toks\fi}
\def\PY@do#1{\PY@bc{\PY@tc{\PY@ul{%
    \PY@it{\PY@bf{\PY@ff{#1}}}}}}}
\def\PY#1#2{\PY@reset\PY@toks#1+\relax+\PY@do{#2}}

\expandafter\def\csname PY@tok@w\endcsname{\def\PY@tc##1{\textcolor[rgb]{0.73,0.73,0.73}{##1}}}
\expandafter\def\csname PY@tok@c\endcsname{\let\PY@it=\textit\def\PY@tc##1{\textcolor[rgb]{0.25,0.50,0.50}{##1}}}
\expandafter\def\csname PY@tok@cp\endcsname{\def\PY@tc##1{\textcolor[rgb]{0.74,0.48,0.00}{##1}}}
\expandafter\def\csname PY@tok@k\endcsname{\let\PY@bf=\textbf\def\PY@tc##1{\textcolor[rgb]{0.00,0.50,0.00}{##1}}}
\expandafter\def\csname PY@tok@kp\endcsname{\def\PY@tc##1{\textcolor[rgb]{0.00,0.50,0.00}{##1}}}
\expandafter\def\csname PY@tok@kt\endcsname{\def\PY@tc##1{\textcolor[rgb]{0.69,0.00,0.25}{##1}}}
\expandafter\def\csname PY@tok@o\endcsname{\def\PY@tc##1{\textcolor[rgb]{0.40,0.40,0.40}{##1}}}
\expandafter\def\csname PY@tok@ow\endcsname{\let\PY@bf=\textbf\def\PY@tc##1{\textcolor[rgb]{0.67,0.13,1.00}{##1}}}
\expandafter\def\csname PY@tok@nb\endcsname{\def\PY@tc##1{\textcolor[rgb]{0.00,0.50,0.00}{##1}}}
\expandafter\def\csname PY@tok@nf\endcsname{\def\PY@tc##1{\textcolor[rgb]{0.00,0.00,1.00}{##1}}}
\expandafter\def\csname PY@tok@nc\endcsname{\let\PY@bf=\textbf\def\PY@tc##1{\textcolor[rgb]{0.00,0.00,1.00}{##1}}}
\expandafter\def\csname PY@tok@nn\endcsname{\let\PY@bf=\textbf\def\PY@tc##1{\textcolor[rgb]{0.00,0.00,1.00}{##1}}}
\expandafter\def\csname PY@tok@ne\endcsname{\let\PY@bf=\textbf\def\PY@tc##1{\textcolor[rgb]{0.82,0.25,0.23}{##1}}}
\expandafter\def\csname PY@tok@nv\endcsname{\def\PY@tc##1{\textcolor[rgb]{0.10,0.09,0.49}{##1}}}
\expandafter\def\csname PY@tok@no\endcsname{\def\PY@tc##1{\textcolor[rgb]{0.53,0.00,0.00}{##1}}}
\expandafter\def\csname PY@tok@nl\endcsname{\def\PY@tc##1{\textcolor[rgb]{0.63,0.63,0.00}{##1}}}
\expandafter\def\csname PY@tok@ni\endcsname{\let\PY@bf=\textbf\def\PY@tc##1{\textcolor[rgb]{0.60,0.60,0.60}{##1}}}
\expandafter\def\csname PY@tok@na\endcsname{\def\PY@tc##1{\textcolor[rgb]{0.49,0.56,0.16}{##1}}}
\expandafter\def\csname PY@tok@nt\endcsname{\let\PY@bf=\textbf\def\PY@tc##1{\textcolor[rgb]{0.00,0.50,0.00}{##1}}}
\expandafter\def\csname PY@tok@nd\endcsname{\def\PY@tc##1{\textcolor[rgb]{0.67,0.13,1.00}{##1}}}
\expandafter\def\csname PY@tok@s\endcsname{\def\PY@tc##1{\textcolor[rgb]{0.73,0.13,0.13}{##1}}}
\expandafter\def\csname PY@tok@sd\endcsname{\let\PY@it=\textit\def\PY@tc##1{\textcolor[rgb]{0.73,0.13,0.13}{##1}}}
\expandafter\def\csname PY@tok@si\endcsname{\let\PY@bf=\textbf\def\PY@tc##1{\textcolor[rgb]{0.73,0.40,0.53}{##1}}}
\expandafter\def\csname PY@tok@se\endcsname{\let\PY@bf=\textbf\def\PY@tc##1{\textcolor[rgb]{0.73,0.40,0.13}{##1}}}
\expandafter\def\csname PY@tok@sr\endcsname{\def\PY@tc##1{\textcolor[rgb]{0.73,0.40,0.53}{##1}}}
\expandafter\def\csname PY@tok@ss\endcsname{\def\PY@tc##1{\textcolor[rgb]{0.10,0.09,0.49}{##1}}}
\expandafter\def\csname PY@tok@sx\endcsname{\def\PY@tc##1{\textcolor[rgb]{0.00,0.50,0.00}{##1}}}
\expandafter\def\csname PY@tok@m\endcsname{\def\PY@tc##1{\textcolor[rgb]{0.40,0.40,0.40}{##1}}}
\expandafter\def\csname PY@tok@gh\endcsname{\let\PY@bf=\textbf\def\PY@tc##1{\textcolor[rgb]{0.00,0.00,0.50}{##1}}}
\expandafter\def\csname PY@tok@gu\endcsname{\let\PY@bf=\textbf\def\PY@tc##1{\textcolor[rgb]{0.50,0.00,0.50}{##1}}}
\expandafter\def\csname PY@tok@gd\endcsname{\def\PY@tc##1{\textcolor[rgb]{0.63,0.00,0.00}{##1}}}
\expandafter\def\csname PY@tok@gi\endcsname{\def\PY@tc##1{\textcolor[rgb]{0.00,0.63,0.00}{##1}}}
\expandafter\def\csname PY@tok@gr\endcsname{\def\PY@tc##1{\textcolor[rgb]{1.00,0.00,0.00}{##1}}}
\expandafter\def\csname PY@tok@ge\endcsname{\let\PY@it=\textit}
\expandafter\def\csname PY@tok@gs\endcsname{\let\PY@bf=\textbf}
\expandafter\def\csname PY@tok@gp\endcsname{\let\PY@bf=\textbf\def\PY@tc##1{\textcolor[rgb]{0.00,0.00,0.50}{##1}}}
\expandafter\def\csname PY@tok@go\endcsname{\def\PY@tc##1{\textcolor[rgb]{0.53,0.53,0.53}{##1}}}
\expandafter\def\csname PY@tok@gt\endcsname{\def\PY@tc##1{\textcolor[rgb]{0.00,0.27,0.87}{##1}}}
\expandafter\def\csname PY@tok@err\endcsname{\def\PY@bc##1{\setlength{\fboxsep}{0pt}\fcolorbox[rgb]{1.00,0.00,0.00}{1,1,1}{\strut ##1}}}
\expandafter\def\csname PY@tok@kc\endcsname{\let\PY@bf=\textbf\def\PY@tc##1{\textcolor[rgb]{0.00,0.50,0.00}{##1}}}
\expandafter\def\csname PY@tok@kd\endcsname{\let\PY@bf=\textbf\def\PY@tc##1{\textcolor[rgb]{0.00,0.50,0.00}{##1}}}
\expandafter\def\csname PY@tok@kn\endcsname{\let\PY@bf=\textbf\def\PY@tc##1{\textcolor[rgb]{0.00,0.50,0.00}{##1}}}
\expandafter\def\csname PY@tok@kr\endcsname{\let\PY@bf=\textbf\def\PY@tc##1{\textcolor[rgb]{0.00,0.50,0.00}{##1}}}
\expandafter\def\csname PY@tok@bp\endcsname{\def\PY@tc##1{\textcolor[rgb]{0.00,0.50,0.00}{##1}}}
\expandafter\def\csname PY@tok@fm\endcsname{\def\PY@tc##1{\textcolor[rgb]{0.00,0.00,1.00}{##1}}}
\expandafter\def\csname PY@tok@vc\endcsname{\def\PY@tc##1{\textcolor[rgb]{0.10,0.09,0.49}{##1}}}
\expandafter\def\csname PY@tok@vg\endcsname{\def\PY@tc##1{\textcolor[rgb]{0.10,0.09,0.49}{##1}}}
\expandafter\def\csname PY@tok@vi\endcsname{\def\PY@tc##1{\textcolor[rgb]{0.10,0.09,0.49}{##1}}}
\expandafter\def\csname PY@tok@vm\endcsname{\def\PY@tc##1{\textcolor[rgb]{0.10,0.09,0.49}{##1}}}
\expandafter\def\csname PY@tok@sa\endcsname{\def\PY@tc##1{\textcolor[rgb]{0.73,0.13,0.13}{##1}}}
\expandafter\def\csname PY@tok@sb\endcsname{\def\PY@tc##1{\textcolor[rgb]{0.73,0.13,0.13}{##1}}}
\expandafter\def\csname PY@tok@sc\endcsname{\def\PY@tc##1{\textcolor[rgb]{0.73,0.13,0.13}{##1}}}
\expandafter\def\csname PY@tok@dl\endcsname{\def\PY@tc##1{\textcolor[rgb]{0.73,0.13,0.13}{##1}}}
\expandafter\def\csname PY@tok@s2\endcsname{\def\PY@tc##1{\textcolor[rgb]{0.73,0.13,0.13}{##1}}}
\expandafter\def\csname PY@tok@sh\endcsname{\def\PY@tc##1{\textcolor[rgb]{0.73,0.13,0.13}{##1}}}
\expandafter\def\csname PY@tok@s1\endcsname{\def\PY@tc##1{\textcolor[rgb]{0.73,0.13,0.13}{##1}}}
\expandafter\def\csname PY@tok@mb\endcsname{\def\PY@tc##1{\textcolor[rgb]{0.40,0.40,0.40}{##1}}}
\expandafter\def\csname PY@tok@mf\endcsname{\def\PY@tc##1{\textcolor[rgb]{0.40,0.40,0.40}{##1}}}
\expandafter\def\csname PY@tok@mh\endcsname{\def\PY@tc##1{\textcolor[rgb]{0.40,0.40,0.40}{##1}}}
\expandafter\def\csname PY@tok@mi\endcsname{\def\PY@tc##1{\textcolor[rgb]{0.40,0.40,0.40}{##1}}}
\expandafter\def\csname PY@tok@il\endcsname{\def\PY@tc##1{\textcolor[rgb]{0.40,0.40,0.40}{##1}}}
\expandafter\def\csname PY@tok@mo\endcsname{\def\PY@tc##1{\textcolor[rgb]{0.40,0.40,0.40}{##1}}}
\expandafter\def\csname PY@tok@ch\endcsname{\let\PY@it=\textit\def\PY@tc##1{\textcolor[rgb]{0.25,0.50,0.50}{##1}}}
\expandafter\def\csname PY@tok@cm\endcsname{\let\PY@it=\textit\def\PY@tc##1{\textcolor[rgb]{0.25,0.50,0.50}{##1}}}
\expandafter\def\csname PY@tok@cpf\endcsname{\let\PY@it=\textit\def\PY@tc##1{\textcolor[rgb]{0.25,0.50,0.50}{##1}}}
\expandafter\def\csname PY@tok@c1\endcsname{\let\PY@it=\textit\def\PY@tc##1{\textcolor[rgb]{0.25,0.50,0.50}{##1}}}
\expandafter\def\csname PY@tok@cs\endcsname{\let\PY@it=\textit\def\PY@tc##1{\textcolor[rgb]{0.25,0.50,0.50}{##1}}}

\def\PYZbs{\char`\\}
\def\PYZus{\char`\_}
\def\PYZob{\char`\{}
\def\PYZcb{\char`\}}
\def\PYZca{\char`\^}
\def\PYZam{\char`\&}
\def\PYZlt{\char`\<}
\def\PYZgt{\char`\>}
\def\PYZsh{\char`\#}
\def\PYZpc{\char`\%}
\def\PYZdl{\char`\$}
\def\PYZhy{\char`\-}
\def\PYZsq{\char`\'}
\def\PYZdq{\char`\"}
\def\PYZti{\char`\~}
% for compatibility with earlier versions
\def\PYZat{@}
\def\PYZlb{[}
\def\PYZrb{]}
\makeatother


    % Exact colors from NB
    \definecolor{incolor}{rgb}{0.0, 0.0, 0.5}
    \definecolor{outcolor}{rgb}{0.545, 0.0, 0.0}



    
    % Prevent overflowing lines due to hard-to-break entities
    \sloppy 
    % Setup hyperref package
    \hypersetup{
      breaklinks=true,  % so long urls are correctly broken across lines
      colorlinks=true,
      urlcolor=urlcolor,
      linkcolor=linkcolor,
      citecolor=citecolor,
      }
    % Slightly bigger margins than the latex defaults
    
    \geometry{verbose,tmargin=1in,bmargin=1in,lmargin=1in,rmargin=1in}
    
    

    \begin{document}
    
    
    \maketitle
    
    

    
    \section{Bode Plot Recipe}\label{bode-plot-recipe}

This notebook will explain the basic recipe to draw Bode Plots for
Circuit Analysis.

    \section{Why a Bode Plot}\label{why-a-bode-plot}

A Bode Plot is a simple graphical representation of the transfer
function of a linear time invariant system. These systems share the
complex exponential function, or the family of sinusoidal functions, as
their characteristic functions. Whenever an input can be represented as
a sum of sinusoidal functions, the transfer function will only affect
the amplitude and the phase of the input. Consequently representing the
transfer function in terms of its magnitude versus frequency and its
phase shift vs frequency fully describes the output of the system. The
two are called Bode Magnitude Plot and Bode Phase Plot.

Usually they are drawn with the frequency in logarithmic scale and the
magnitude in units of decibel: \(20 \cdot log_{10} (H(\omega))\) and the
phase \(\phi\) in linear scale.

This is a useful tool because LTI systems all derive from ordinary
differential equations in the time domain, which, in turn, lead to
transfer functions in the frequency domain that can be expressed in
terms of a fraction of two polynomials. More specific, as a fraction of
the form

\(H(j \omega) = K \cdot \frac{ \prod_{i=1}^{M}j\omega - a_i}{\prod_{i=1}^{N} j\omega -b_i}\)

with the system attenuation then given as

\(a(j\omega) = 20 \cdot log(|H(j\omega)|) = 20 \cdot log(K) + \sum_{i=1}^{M} 20 \cdot log(j\omega-a_i) + \sum_{i=1}^{M} 20 \cdot log(\frac{1}{ j\omega-b_i)}\)

and the phase curve is

\(\phi(j\omega) = \sum_{i=1}^{M} arg(j\omega-a_i) + \sum_{i=1}^{M} arg(\frac{1}{ j\omega-b_i)}\)

We can work with these roots and zeroes because they remain after the
generation of the absolute value. For example

\(\big| \frac{K(-a+iw)(-b+iw)}{(-c+iw)(-d+iw)}\big|\) simplifies to

\(\frac{\sqrt{K^2}\sqrt{a^2+w^2}\sqrt{b^2+w^2}}{\sqrt{c^2+w^2}\sqrt{d^2+w^2}}\).

Consequently we can build each systems frequency response from summing
the freqency responses of individual poles and zeros of the system. What
we need for this are the basic plots, for single poles and zeros.

    \begin{Verbatim}[commandchars=\\\{\}]
{\color{incolor}In [{\color{incolor}1}]:} \PY{k+kn}{from} \PY{n+nn}{scipy} \PY{k}{import} \PY{n}{signal}
        \PY{o}{\PYZpc{}}\PY{k}{matplotlib} inline
        \PY{k+kn}{import} \PY{n+nn}{matplotlib}\PY{n+nn}{.}\PY{n+nn}{pyplot} \PY{k}{as} \PY{n+nn}{plt}
        
        
        \PY{n}{s1} \PY{o}{=} \PY{n}{signal}\PY{o}{.}\PY{n}{lti}\PY{p}{(}\PY{l+m+mi}{1}\PY{p}{,}\PY{p}{[}\PY{l+m+mi}{1}\PY{p}{,}\PY{l+m+mi}{10}\PY{p}{]}\PY{p}{)}   \PY{c+c1}{\PYZsh{} System will have a numerator polynomial of 1 and a denominator of 1*w + 10}
        \PY{n}{w}\PY{p}{,} \PY{n}{mag}\PY{p}{,} \PY{n}{phase} \PY{o}{=} \PY{n}{signal}\PY{o}{.}\PY{n}{bode}\PY{p}{(}\PY{n}{s1}\PY{p}{)}
        \PY{n}{plt}\PY{o}{.}\PY{n}{figure}\PY{p}{(}\PY{p}{)}
        \PY{n}{plt}\PY{o}{.}\PY{n}{semilogx}\PY{p}{(}\PY{n}{w}\PY{p}{,} \PY{n}{mag}\PY{p}{)}    \PY{c+c1}{\PYZsh{} Bode magnitude plot}
        \PY{n}{plt}\PY{o}{.}\PY{n}{figure}\PY{p}{(}\PY{p}{)}
        \PY{n}{plt}\PY{o}{.}\PY{n}{semilogx}\PY{p}{(}\PY{n}{w}\PY{p}{,} \PY{n}{phase}\PY{p}{)}  \PY{c+c1}{\PYZsh{} Bode phase plot}
        \PY{n}{plt}\PY{o}{.}\PY{n}{show}\PY{p}{(}\PY{p}{)}
\end{Verbatim}


    \begin{center}
    \adjustimage{max size={0.9\linewidth}{0.9\paperheight}}{output_2_0.png}
    \end{center}
    { \hspace*{\fill} \\}
    
    \begin{center}
    \adjustimage{max size={0.9\linewidth}{0.9\paperheight}}{output_2_1.png}
    \end{center}
    { \hspace*{\fill} \\}
    
    The plot for the single pole can be derived from
\(H(j\omega) = K\frac{1}{(j\omega-b)}\) at its characteristic values. We
always take the logarithm of the absolute value of \(H\), so for a pole:
\(|H(jw)| = \sqrt{H \cdot H^\ast}\). Specifically at any place where
there is a zero or pole with \((jw+/-b)\) the magnitude of this term
there is \(\sqrt{(j\omega+/-b)(-j\omega+/-b)}= \sqrt{b^2+\omega^2}\)
(for the zero, the inverse for the pole). This is important, as the root
in the complex plane appears for positive or negative frequency, but the
magnitude is always finite there! Taking the limit
\(\omega \rightarrow 0\) we arrive at \$20 log(H(j\omega)) = 20 log( K
\frac{1}{-b}) = 20 log(K) + 20 log(b) \$. For the limit
\(\omega \rightarrow \infty\) we have
\(20 log(H(j\omega)) = 20 log(K) - 20 log(j \omega) = -\infty\).And, as
seen above, at the place of the pole the magnitude is
\(20 log(\frac{1}{\sqrt{b^2+b^2}}) = - 20 log(\sqrt{2b^2})\) In the
above case for \(b = 10\) and \(K = 1\) we have
\(- 20 \cdot 0.5 \cdot log(2 b^2) = - 10 log(2) - 20 log (b) = - 10 log (2) - 20 log (10) =-3 - 20 = -23 dB\)

    \begin{Verbatim}[commandchars=\\\{\}]
{\color{incolor}In [{\color{incolor}2}]:} \PY{n}{s1} \PY{o}{=} \PY{n}{signal}\PY{o}{.}\PY{n}{lti}\PY{p}{(}\PY{p}{[}\PY{l+m+mi}{10}\PY{p}{,}\PY{l+m+mi}{0}\PY{p}{]}\PY{p}{,}\PY{p}{[}\PY{l+m+mi}{1}\PY{p}{,}\PY{l+m+mi}{10}\PY{p}{]}\PY{p}{)}   \PY{c+c1}{\PYZsh{} System will have a numerator polynomial of 10*w and a denominator of 1*w + 10}
        \PY{n}{w}\PY{p}{,} \PY{n}{mag}\PY{p}{,} \PY{n}{phase} \PY{o}{=} \PY{n}{signal}\PY{o}{.}\PY{n}{bode}\PY{p}{(}\PY{n}{s1}\PY{p}{)}
        \PY{n}{plt}\PY{o}{.}\PY{n}{figure}\PY{p}{(}\PY{p}{)}
        \PY{n}{plt}\PY{o}{.}\PY{n}{semilogx}\PY{p}{(}\PY{n}{w}\PY{p}{,} \PY{n}{mag}\PY{p}{)}    \PY{c+c1}{\PYZsh{} Bode magnitude plot}
        \PY{n}{plt}\PY{o}{.}\PY{n}{figure}\PY{p}{(}\PY{p}{)}
        \PY{n}{plt}\PY{o}{.}\PY{n}{semilogx}\PY{p}{(}\PY{n}{w}\PY{p}{,} \PY{n}{phase}\PY{p}{)}  \PY{c+c1}{\PYZsh{} Bode phase plot}
        \PY{n}{plt}\PY{o}{.}\PY{n}{show}\PY{p}{(}\PY{p}{)}
\end{Verbatim}


    \begin{center}
    \adjustimage{max size={0.9\linewidth}{0.9\paperheight}}{output_4_0.png}
    \end{center}
    { \hspace*{\fill} \\}
    
    \begin{center}
    \adjustimage{max size={0.9\linewidth}{0.9\paperheight}}{output_4_1.png}
    \end{center}
    { \hspace*{\fill} \\}
    
    Finding the starting values for the amplitude plot:

The initial value of the graph depends on the boundaries. The initial
point is found by putting the initial angular frequency \$ \omega\$ into
the function and finding \$
\textbar{}H(\{\mathrm  {j}\}\omega )\textbar{}\$

For the changing of slopes we follow these rules:

\begin{itemize}
\tightlist
\item
  at every value of s where \(\omega =a_{n}\) (a zero), increase the
  slope of the line by \$20,\mathrm {dB} \$ per decade. Do this multiple
  times if the zero is present more than once.
\item
  at every value of s where\$ \omega =b\_\{n\}\$ (a pole), decrease the
  slope of the line by \$ 20,\mathrm {dB}\$ per decade. Do this multiple
  times if the zero is present more than once.
\item
  The initial slope of the function at the initial value depends on the
  number and order of zeros and poles that are at values below the
  initial value, and are found using the first two rules.
\end{itemize}

For zeros and poles at zero frequency, represented as \$j\omega \$ or
\(\frac{1}{j\omega}\) we use the magnitude at \(\omega = 1 rad/s\) which
is equal to the DC Magnitude \(K\) as \$\textbar{} H(j\omega)\textbar{}
= K \sqrt{\omega^2} \$ (see above) and that is equal to K at
\(\omega = 1\).

Finding the starting values for the phase plot:

\begin{itemize}
\tightlist
\item
  Start with a horizontal line at \(\phi = 0\) if the gain is positive
  and at \(\phi = -180\) if the gain is negative (that is an inverted
  sinusoid which is corresponding to a 180 degree phase shift between
  input and output.
\item
  For each zero decrease the slope by 45 degrees/decade one decade
  before the zero and and increase the slope by 45 degrees one decade
  after the zero. Thus, after two decades the phase will be -90 with
  respect to before the zero.
\item
  For each pole do the inverse.
\item
  For a pole or zero at \(\omega = 0\) we will add -/+ 90 degrees from
  the start (we never see zero, so the argument is not indeterminate and
  leads to -90 for a pole and 90 for a zero)
\end{itemize}

    \section{The Effect of sign changes}\label{the-effect-of-sign-changes}

In order to see what effect a change in sign of the frequency of the
zero we get, lets plot two plots in the same picture. One with a
positive frequency zero / pole and one with a negative one.

    \begin{Verbatim}[commandchars=\\\{\}]
{\color{incolor}In [{\color{incolor}3}]:} \PY{n}{s1} \PY{o}{=} \PY{n}{signal}\PY{o}{.}\PY{n}{lti}\PY{p}{(}\PY{l+m+mi}{1}\PY{p}{,}\PY{p}{[}\PY{l+m+mi}{1}\PY{p}{,}\PY{l+m+mi}{10}\PY{p}{]}\PY{p}{)}   \PY{c+c1}{\PYZsh{} System will have a numerator polynomial of 1 and a denominator of 1*w + 10}
        \PY{n}{s2} \PY{o}{=} \PY{n}{signal}\PY{o}{.}\PY{n}{lti}\PY{p}{(}\PY{l+m+mi}{1}\PY{p}{,}\PY{p}{[}\PY{l+m+mi}{1}\PY{p}{,}\PY{o}{\PYZhy{}}\PY{l+m+mi}{10}\PY{p}{]}\PY{p}{)}  \PY{c+c1}{\PYZsh{} System will have a numerator polynomial of 1 and a denominator of 1*w \PYZhy{} 10}
        \PY{n}{w1}\PY{p}{,} \PY{n}{mag1}\PY{p}{,} \PY{n}{phase1} \PY{o}{=} \PY{n}{signal}\PY{o}{.}\PY{n}{bode}\PY{p}{(}\PY{n}{s1}\PY{p}{)}
        \PY{n}{w2}\PY{p}{,} \PY{n}{mag2}\PY{p}{,} \PY{n}{phase2} \PY{o}{=} \PY{n}{signal}\PY{o}{.}\PY{n}{bode}\PY{p}{(}\PY{n}{s2}\PY{p}{)}
        \PY{n}{plt}\PY{o}{.}\PY{n}{figure}\PY{p}{(}\PY{p}{)}
        \PY{n}{plt}\PY{o}{.}\PY{n}{semilogx}\PY{p}{(}\PY{n}{w1}\PY{p}{,} \PY{n}{mag1}\PY{p}{)}    \PY{c+c1}{\PYZsh{} Bode magnitude plot}
        \PY{n}{plt}\PY{o}{.}\PY{n}{semilogx}\PY{p}{(}\PY{n}{w2}\PY{p}{,}\PY{n}{mag2}\PY{p}{)}
        \PY{n}{plt}\PY{o}{.}\PY{n}{figure}\PY{p}{(}\PY{p}{)}
        \PY{n}{plt}\PY{o}{.}\PY{n}{semilogx}\PY{p}{(}\PY{n}{w1}\PY{p}{,} \PY{n}{phase1}\PY{p}{)}  \PY{c+c1}{\PYZsh{} Bode phase plot}
        \PY{n}{plt}\PY{o}{.}\PY{n}{semilogx}\PY{p}{(}\PY{n}{w2}\PY{p}{,} \PY{n}{phase2}\PY{p}{)}
        \PY{n}{plt}\PY{o}{.}\PY{n}{show}\PY{p}{(}\PY{p}{)}
\end{Verbatim}


    \begin{center}
    \adjustimage{max size={0.9\linewidth}{0.9\paperheight}}{output_7_0.png}
    \end{center}
    { \hspace*{\fill} \\}
    
    \begin{center}
    \adjustimage{max size={0.9\linewidth}{0.9\paperheight}}{output_7_1.png}
    \end{center}
    { \hspace*{\fill} \\}
    
    \begin{Verbatim}[commandchars=\\\{\}]
{\color{incolor}In [{\color{incolor}4}]:} \PY{n}{As} \PY{n}{we} \PY{n}{can} \PY{n}{see} \PY{n}{above}\PY{p}{,} \PY{n}{the} \PY{n}{magnitude} \PY{n}{plot} \PY{n}{does} \PY{o+ow}{not} \PY{n}{change} \PY{k}{with} \PY{n}{a} \PY{n}{change} \PY{o+ow}{in} \PY{n}{sign}\PY{o}{.} 
        \PY{n}{That} \PY{o+ow}{is} \PY{n}{due} \PY{n}{to} \PY{n}{the} \PY{n}{fact} \PY{n}{that} \PY{k}{for} \PY{n}{the} \PY{n}{magnitude} \PY{n}{we} \PY{n}{always} \PY{n}{get} \PY{n}{squares} \PY{k}{for} \PY{n}{the} \PY{n}{frequency} \PY{o+ow}{and} \PY{n}{the} \PY{n}{offset}\PY{o}{.} 
        \PY{n}{However}\PY{p}{,} \PY{o+ow}{in} \PY{n}{the} \PY{n}{phase} \PY{n}{plot}\PY{p}{,} \PY{n}{we} \PY{n}{need} \PY{n}{to} \PY{n}{differentiate} \PY{n}{between} \PY{n}{the} \PY{n}{minus} \PY{o+ow}{and} \PY{n}{plus}\PY{o}{.} \PY{n}{Here} \PY{n}{we} \PY{n}{get} \PY{n}{a} \PY{n}{difference} \PY{o+ow}{in} 
        \PY{n}{sign} \PY{n}{of} \PY{n}{the} \PY{n}{slope} \PY{o+ow}{and} \PY{n}{a} \PY{n}{phase} \PY{n}{shift} \PY{n}{of} \PY{l+m+mi}{180} \PY{n}{degree}\PY{o}{.} 
\end{Verbatim}


    \begin{Verbatim}[commandchars=\\\{\}]

          File "<ipython-input-4-038ecfebcd7a>", line 1
        As we can see above, the magnitude plot does not change with a change in sign.
            \^{}
    SyntaxError: invalid syntax
    

    \end{Verbatim}

    \begin{Verbatim}[commandchars=\\\{\}]
{\color{incolor}In [{\color{incolor}5}]:} \PY{n}{s1} \PY{o}{=} \PY{n}{signal}\PY{o}{.}\PY{n}{lti}\PY{p}{(}\PY{p}{[}\PY{l+m+mi}{1}\PY{p}{,}\PY{l+m+mi}{5}\PY{p}{]}\PY{p}{,}\PY{p}{[}\PY{l+m+mi}{1}\PY{p}{,}\PY{l+m+mi}{10}\PY{p}{]}\PY{p}{)}   \PY{c+c1}{\PYZsh{} System will have a numerator polynomial of 1*w +5 and a denominator of 1*w + 10}
        \PY{n}{s2} \PY{o}{=} \PY{n}{signal}\PY{o}{.}\PY{n}{lti}\PY{p}{(}\PY{p}{[}\PY{l+m+mi}{1}\PY{p}{,}\PY{o}{\PYZhy{}}\PY{l+m+mi}{5}\PY{p}{]}\PY{p}{,}\PY{p}{[}\PY{l+m+mi}{1}\PY{p}{,}\PY{o}{\PYZhy{}}\PY{l+m+mi}{10}\PY{p}{]}\PY{p}{)}  \PY{c+c1}{\PYZsh{} System will have a numerator polynomial of 1*w \PYZhy{}5 and a denominator of 1*w \PYZhy{} 10}
        \PY{n}{w1}\PY{p}{,} \PY{n}{mag1}\PY{p}{,} \PY{n}{phase1} \PY{o}{=} \PY{n}{signal}\PY{o}{.}\PY{n}{bode}\PY{p}{(}\PY{n}{s1}\PY{p}{)}
        \PY{n}{w2}\PY{p}{,} \PY{n}{mag2}\PY{p}{,} \PY{n}{phase2} \PY{o}{=} \PY{n}{signal}\PY{o}{.}\PY{n}{bode}\PY{p}{(}\PY{n}{s2}\PY{p}{)}
        \PY{n}{plt}\PY{o}{.}\PY{n}{figure}\PY{p}{(}\PY{p}{)}
        \PY{n}{plt}\PY{o}{.}\PY{n}{semilogx}\PY{p}{(}\PY{n}{w1}\PY{p}{,} \PY{n}{mag1}\PY{p}{)}    \PY{c+c1}{\PYZsh{} Bode magnitude plot}
        \PY{n}{plt}\PY{o}{.}\PY{n}{semilogx}\PY{p}{(}\PY{n}{w2}\PY{p}{,}\PY{n}{mag2}\PY{p}{)}
        \PY{n}{plt}\PY{o}{.}\PY{n}{figure}\PY{p}{(}\PY{p}{)}
        \PY{n}{plt}\PY{o}{.}\PY{n}{semilogx}\PY{p}{(}\PY{n}{w1}\PY{p}{,} \PY{n}{phase1}\PY{p}{)}  \PY{c+c1}{\PYZsh{} Bode phase plot}
        \PY{n}{plt}\PY{o}{.}\PY{n}{semilogx}\PY{p}{(}\PY{n}{w2}\PY{p}{,} \PY{n}{phase2}\PY{p}{)}
        \PY{n}{plt}\PY{o}{.}\PY{n}{show}\PY{p}{(}\PY{p}{)}
\end{Verbatim}


    \begin{center}
    \adjustimage{max size={0.9\linewidth}{0.9\paperheight}}{output_9_0.png}
    \end{center}
    { \hspace*{\fill} \\}
    
    \begin{center}
    \adjustimage{max size={0.9\linewidth}{0.9\paperheight}}{output_9_1.png}
    \end{center}
    { \hspace*{\fill} \\}
    
    \section{Rules for drawing asymptotic Bode
Plots}\label{rules-for-drawing-asymptotic-bode-plots}

\begin{enumerate}
\def\labelenumi{\arabic{enumi}.}
\tightlist
\item
  Rearrange the tranfer function in the following format:
  \(H(s) = A\frac{(s/z_0 + 1)(s/z_1 + 1)\cdots(s/z_n + 1)}{(s/p_0 + 1)(s/p_1 + 1)\cdots(s/p_n + 1)}\)
  where \(A\) is the gain of the system, \(z_0\),\(z_1\),...,\(z_n\) are
  the location of the zeros and \(p_0\),\(p_1\),...,\(p_n\) are the
  location of the poles. The poles and zeros can be in the left hand
  plane (LHP) or right hand plane (RHP). Each Pole or Zero has its
  frequency at \(s= z\) or \(s=p\). For plotting in \(\omega\) we then
  chose \(s=j\omega\). If we divide the range of values for the
  poles/zero between negative and positive, the LHP has the negative
  poles/zeros and the RHP has the positive poles/zeros.
\end{enumerate}

\subsection{Rules for the magnitude
plot}\label{rules-for-the-magnitude-plot}

\begin{enumerate}
\def\labelenumi{\arabic{enumi}.}
\setcounter{enumi}{1}
\tightlist
\item
  The plot starts with a horizontal line at a magnitude equal to the DC
  magnitude of the system \(H(0) = A\). This works only if there is no
  pole or zero at \(\omega = 0\), please see below for that
\end{enumerate}

\begin{enumerate}
\def\labelenumi{\arabic{enumi}.}
\setcounter{enumi}{2}
\tightlist
\item
  For every pole, the slope of the line decreases by 20 dB/decade of
  frequency at that pole's frequency. For the magnitude plot we do not
  have to distinguish between Lefthand Plane or Righthand Plane Poles or
  Zeros.
\end{enumerate}

\begin{enumerate}
\def\labelenumi{\arabic{enumi}.}
\setcounter{enumi}{3}
\tightlist
\item
  For every zero, the slope of the line increases by 20 dB/decade of
  frequency at that zero's frequency
\end{enumerate}

\begin{enumerate}
\def\labelenumi{\arabic{enumi}.}
\setcounter{enumi}{4}
\tightlist
\item
  For every pole or zero at zero frequency, the plot starts with the
  effect of that pole/zero on the slope. Zeros and poles at zero
  frequency are represented as : \(H(s) = A\cdot s\) (Zero) and
  \(H(s) = A \frac{1}{s}\) (pole). This means that at frequency
  \(1 rad/s\), the magnitude must be equal to the DC magnitude A. Then,
  the trace of the plot must cross A at 1 rad/s and be backtracked to
  the starting frequency of the plot.
\end{enumerate}

\begin{enumerate}
\def\labelenumi{\arabic{enumi}.}
\setcounter{enumi}{5}
\tightlist
\item
  For multiple zeros or poles at the same frequency, the slope of the
  line changes according to that number
\end{enumerate}

\subsection{Rules for the Phase Plot}\label{rules-for-the-phase-plot}

\begin{enumerate}
\def\labelenumi{\arabic{enumi}.}
\tightlist
\item
  Start the plot with a horizontal line at 0º phase if the gain is
  positive or -180º if the gain (A) is negative (negative gain
  corresponds to an inverted sinusoid and thus to a 180º phase between
  input and output)
\end{enumerate}

\begin{enumerate}
\def\labelenumi{\arabic{enumi}.}
\setcounter{enumi}{1}
\tightlist
\item
  As the phase depends on the sign of \(\frac{Im(H)}{Re(H)}\) we now
  have to distinguish LHP and RHP poles and zeros. For each LHP pole or
  RHP zero, decrease the slope by 45º/decade one decade before the
  pole/zero, and increase by the same amount one decade after the
  pole/zero. After two decades, the phase will be -90º than before for
  each pole/zero
\end{enumerate}

\begin{enumerate}
\def\labelenumi{\arabic{enumi}.}
\setcounter{enumi}{2}
\tightlist
\item
  For each RHP pole or LHP zero, increase the slope by 45º/decade one
  decade before the pole/zero, and decrease by the same amount one
  decade after the pole/zero. After two decades, the phase will be +90º
  than before for each pole/zero.
\end{enumerate}

\begin{enumerate}
\def\labelenumi{\arabic{enumi}.}
\setcounter{enumi}{3}
\tightlist
\item
  For each pole or zero at zero frequency, the plot starts with the
  effect of that pole/zero on the phase.
\end{enumerate}

\begin{itemize}
\tightlist
\item
  For a pole:
  \(\angle H(s) = \angle \frac{1}{s} = \angle \frac{1}{j\omega} = \angle -\frac{j}{\omega} = \tan^{-1} -\frac{1/\omega}{0} = \tan^{-1}{-\infty} = -90º\).
\item
  And for a Zero:
  \(\angle H(s) = \angle s = \angle j\omega = \tan^{-1} \frac{\omega}{0} = \tan^{-1}{\infty} = 90º\).
  Since bode plots are drawn in logarithmic scale, we never see the zero
  frequency. Hence, for frequencies just above zero, the ratio above for
  the zero is not indeterminate and leads to 90º.
\end{itemize}

\subsection{Transferfunction Plots}\label{transferfunction-plots}

You can plot transferfunctions without the approximations. The magnitude
of the system is taken from the absolute value of the magnitude:

\(|H(s)| = \left|A\frac{(s/z_0 + 1)(s/z_1 + 1)\cdots(s/z_n + 1)}{(s/p_0 + 1)(s/p_1 + 1)\cdots(s/p_n + 1)} \right|\)
\(|H(s)| = |A|\sqrt{\frac{((\omega/z_0)^2 + 1)((\omega/z_1)^2 + 1)\cdots((\omega/z_n)^2 + 1)}{((\omega/p_0)^2 + 1)((\omega/z_1)^2 + 1)\cdots((\omega/z_n)^2 + 1)}}\)

These will differ from the asymptotic plots mainly around the pole and
zero frequencies, where the magnitude error is 3dB.

The phase of the system is taken from the contribution of each pole and
zero for the total phase.

\(\angle H(s) = \angle A - \tan^{-1}(\frac{\omega}{z_0}) - \tan^{-1}(\frac{\omega}{z_1}) - \cdots - \tan^{-1}(\frac{\omega}{z_n}) + \tan^{-1}(\frac{\omega}{p_0}) + \tan^{-1}(\frac{\omega}{p_1}) + \cdots + \tan^{-1}(\frac{\omega}{p_n})\)

The phase transitions will be mainly wrong one decade before and after
the pole / zero frequency.

    License:

This notebook has been compiled by Prof. Dr. Markus Pfeil at the
Hochschule Ravensburg Weingarten. The Pictures for the Plotting Rules
are used from http://www.onmyphd.com/?p=bode.plot and fair usage is
claimed.

Dieses Werk ist lizenziert unter einer Creative Commons Namensnennung -
Weitergabe unter gleichen Bedingungen 4.0 International Lizenz.


    % Add a bibliography block to the postdoc
    
    
    
    \end{document}
